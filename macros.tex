\def\To{\Rightarrow}

\def\cate#1{\textsf{#1}}

\def\ran{\mathrm{ran}}
\def\Ran{\mathrm{Ran}}
\def\RAN{\textsc{Ran}}
\def\lan{\mathrm{lan}}
\def\Lan{\mathrm{Lan}}
\def\LAN{\textsc{Lan}}
\def\Yan{\mathrm{Yan}}
\def\ryft{\mathrm{rift}}
\def\Ryft{\mathrm{Rift}}
\def\RYFT{\textsc{Rift}}
\def\lyft{\mathrm{lift}}
\def\Lyft{\mathrm{Lift}}
\def\LYFT{\textsc{Lift}}
\def\Nat{\textsf{Nat}}
% ^ TODO: these two chunks can be done procedurally and extended


\newcommand{\arXivPreprint}[1]{\href{http://arxiv.org/abs/#1}{arXiv:#1} preprint}

\newlength{\seplen}
\setlength{\seplen}{5pt}
\newlength{\sepwid}
\setlength{\sepwid}{.4pt}
\def\firstblank{\,\rule{\seplen}{\sepwid}\,}
\def\sndblank{\firstblank\llap{\raisebox{2pt}{\firstblank}}}


\def\op{\text{op}}
\def\co{\text{co}}
\def\coop{\text{coop}}

\newcommand{\celtag}[2][dr]{\ar[#1,white, "#2"{black,description}]}


\let\xto\xrightarrow
\let\xot\xleftarrow

\newcommand{\rot}[3][c]{\rotatebox[origin=#1]{#2}{$#3$}}
\newcommand{\rotarrow}[2][c]{\rot{#1}{#2}{\Rightarrow}}

\newcommand{\Nearrow}{\rotarrow{45}}
\newcommand{\Nwarrow}{\rotarrow{135}}
\newcommand{\Searrow}{\rotarrow{-45}}
\newcommand{\Swarrow}{\rotarrow{225}}
\newcommand{\Sarrow}{\rotarrow{-90}}
\newcommand{\Narrow}{\rotarrow{90}}

\DeclareMathOperator{\colim}{colim}
\DeclareMathOperator{\coker}{coker}
\newcommand{\bsmat}{\begin{smallmatrix}}
\newcommand{\esmat}{\end{smallmatrix}}

\makeatletter
  \def\@cite#1#2{[\textsf{#1}\if@tempswa , #2\fi]}
  \def\@biblabel#1{[\textsf{#1}]}
\makeatother


\newcommand{\pto}{\mathop{\ooalign{\hfil\kern-2pt$\mapstochar$\hfil\cr\hfil$\to$\hfil}}}

\usepackage{turnstile}
\newcommand{\adjunct}[2]{\nsststile{#2}{#1}}


\newcommand{\putbib}[2]{
  \bibliography{#1}{}
  \bibliographystyle{#2}
}

\usetikzlibrary{decorations.markings}
\tikzstyle{pro} =
  [ decoration={ markings
               , mark = at position 0.5 with
                   {\draw[-] (0,-1.25pt) -- (0,1.25pt);}
               }
  , postaction ={decorate}
  ]

\DeclareMathAlphabet{\mybb}{U}{BOONDOX-ds}{m}{n}

\newcommand*{\Scale}[2][4]{\scalebox{#1}{\ensuremath{#2}}}%
\def\lrc{
\begin{tikzpicture}[scale=.33]
	\draw[-] (0,0) -| (1,1);
\end{tikzpicture}
}

\def\dopb{\begin{tikzpicture}[scale=.375] \draw (0,0) -| (1,1);\end{tikzpicture}}
\newcommand{\xpb}[1][dr]{\ar@{}[#1]|(.375){\dopb}}
\def\dopo{\begin{tikzpicture}[scale=.375] \draw (0,0) |- (1,1);\end{tikzpicture}}
\newcommand{\xpo}[1][dr]{\ar@{}[#1]|(.675){\dopo}}

\newcommand{\pb}{\arrow[dr, phantom, "\dopb", very near start]}
\newcommand{\po}{\arrow[dr, phantom, "\dopo", very near end]}

\def\[{\begin{equation}}
\def\]{\end{equation}}
% ^ this is BAD, but can't do without at this point...

\newenvironment{xsmallmatrix}[1]
{\renewcommand\thickspace{\kern#1}\smallmatrix}
{\endsmallmatrix}
\NewDocumentCommand{\prevar}{o m m}{
	\IfNoValueTF{#1}{
			\begin{smallmatrix}
				#2 \\
				\downarrow \\
				#3
			\end{smallmatrix}}
	{
			\begin{xsmallmatrix}{0em}
				& #2 \\
				#1 & \downarrow \\
				& #3
			\end{xsmallmatrix}}}
\NewDocumentCommand{\var}{o m m}{
	\IfNoValueTF{#1}{
		\left[
			\begin{smallmatrix}
				#2 \\
				\downarrow \\
				#3
			\end{smallmatrix}\right]}
	{
		\left[
			\begin{xsmallmatrix}{0em}
				& #2 \\
				#1 & \downarrow \\
				& #3
			\end{xsmallmatrix}\right]}}

\usepackage{adjustbox}
\newenvironment{adju}[1][0.925]{%
\begin{center}\begin{adjustbox}{max height=0.5\textheight, max width=#1\textwidth}}{\end{adjustbox}\end{center}}

\newcommand{\precop}[2]{
  \begin{smallmatrix}
    #1 \\ #2
  \end{smallmatrix}
}
\newcommand{\cop}[2]{
  \left[\begin{smallmatrix}
    #1 \\ #2
  \end{smallmatrix}\right]
}
\def\emdash{\text{-}}

\newcommand{\adja}[4]{
  #1 : \xymatrix{#2 \ar@<.5em>[r] \ar@{}[r]|\perp & \ar@<.5em>[l] #3 } : #4
}


\NewDocumentCommand{\vxy}{o m}{
  \IfNoValueTF{#1}
  {\vcenter{\xymatrix{#2}}} % do if \vxy{bla}
  {\vcenter{\xymatrix#1{#2}}} % do if \vxy[foo]{bla}
}